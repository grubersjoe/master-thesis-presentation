\documentclass{beamer}

% Metadata
\title{Statische Typsysteme für JavaScript}
\subtitle{Entwicklung eines Transpilers zur Übersetzung von Flow nach TypeScript}
\author{Jonathan Gruber}
\institute{HTWK Leipzig}
\date{18. Dezember 2019}

% Encoding
\usepackage[utf8]{inputenc}
\usepackage[T1]{fontenc}

% Babel
\usepackage[ngerman]{babel}

% Fonts
\usepackage{arev} % for Math
\usepackage[sb]{plex-sans}
\usepackage[scale=0.8,sb]{plex-mono}

% Biber (references)
\usepackage[backend=biber]{biblatex}
\addbibresource{src/references.bib}

\let\OLDitemize\itemize
\renewcommand\itemize{\OLDitemize\addtolength{\itemsep}{4pt}}

\let\OLDenumerate\enumerate
\renewcommand\enumerate{\OLDenumerate\addtolength{\itemsep}{2pt}}

\usepackage{csquotes}
\usepackage{longtable, tabu, booktabs}
\usepackage{tikz}
\usepackage{hyperref}

% Code listings
\usepackage{listings}

\lstset{
  aboveskip=0pt,
  basicstyle=\linespread{1.04}\small\ttfamily,
  belowcaptionskip=0pt,
  belowskip=0pt,
  breakatwhitespace=true,
  breaklines=true,
  comment=[l]{//},
  commentstyle=\color{darkgray},
  emph={?, any, Array, boolean, empty, interface, mixed, never, number, null, opaque, string, this, unknown, void, T, TypeAlias},
  emphstyle=\itshape,
  extendedchars=false,
  keywords={abstract, arguments, await, boolean, break, byte, case, catch, char, class, const, continue, debugger, declare, default, delete, do, double, else, enum, eval, export, extends, false, final, finally, float, for, function, goto, if, implements, import, in, instanceof, int, let, long, module, native, new, of, package, private, protected, public, return, short, static, super, switch, synchronized, this, throw, throws, transient, true, try, typeof, var, volatile, while, with, yield, type},
  keywordstyle=\bfseries,
  morecomment=[s]{/*}{*/},
  morestring=[b]',
  morestring=[b]",
  numbers=left,
  numbersep=12pt,
  numberstyle=\linespread{1.04}\scriptsize\ttfamily\color{gray},
  showspaces=false,
  showstringspaces=false,
  stepnumber=1,
  stringstyle=\color{black},
  tabsize=4,
}

% The following ensures that only straight quotes are used inside of \texttt
\usepackage{upquote}
\usepackage{etoolbox}

\robustify{\texttt}
\let\originaltexttt\texttt

\begingroup
\catcode`'=\active
\catcode``=\active
\globaldefs1
\makeatletter
\renewrobustcmd{\texttt}[1]{%
  {%
  \everyeof{\noexpand}\endlinechar-1
  \expandafter\catcode\string``=\active
  \expandafter\catcode\string`'=\active
  \let'\textquotesingle
  \let`\textasciigrave
  \ifx\encodingdefault\upquote@OTone
    \ifx\ttdefault\upquote@cmtt
    \def'{\char13 }\def`{\char18 }%
    \fi
  \fi
  \scantokens{\originaltexttt{#1}}%
  }%
}%
\endgroup


% Trees
\usepackage{forest,subcaption}

% Style for edge labels
\tikzset{edge/.style = {
  midway,fill=white,font=\scriptsize
}}

\forestset{
  font=\footnotesize,
  default preamble={
    for tree = {
      l=1.5cm,
      s sep=.75cm,
      inner xsep=2mm,
      inner ysep=1.25mm,
      minimum height=4.75mm,
      rectangle,
      rounded corners=2,
      fill=mlightgray,
      font=\footnotesize,
    }
  }
}


% Theme
\definecolor{pcolor}{HTML}{06594f}
\definecolor{scolor}{HTML}{01517f}
\definecolor{mlightgray}{HTML}{DCDCDC}
\usetheme{Frankfurt}
\setbeamercolor{bibliography entry author}{fg=black}
\setbeamercolor{bibliography entry location}{fg=black}
\setbeamercolor{bibliography entry note}{fg=black}
\setbeamercolor{bibliography entry title}{fg=black}
\setbeamercolor{enumerate item}{fg=black}
\setbeamercolor{enumerate subitem}{fg=black}
\setbeamercolor{itemize item}{fg=black}
\setbeamercolor{itemize subitem}{fg=black}
\setbeamerfont{frametitle}{series=\bfseries,parent=structure}
\setbeamertemplate{bibliography item}{\insertbiblabel}
\setbeamertemplate{blocks}[rounded][shadow=false]
\setbeamertemplate{enumerate items}[default]
\setbeamertemplate{frametitle}[default][shadow=false]
\setbeamertemplate{footline}[frame number]
\setbeamertemplate{itemize items}[circle]
\setbeamertemplate{navigation symbols}{}
\setbeamertemplate{title page}[default][rounded=true,shadow=false]
\usecolortheme[named=pcolor]{structure}

% Small font in bibliography
\renewcommand*{\bibfont}{\small}

% Titlepage
\makeatletter
\setbeamertemplate{title page}
{
  \vfill
  \begin{centering}
    \setbeamercolor{title}{bg=white,fg=black}

    \begin{beamercolorbox}[center]{author}
      \usebeamerfont{author}\normalsize Masterkolloquium
    \end{beamercolorbox}

    \vspace{0.5cm}

    \begin{beamercolorbox}[sep=8pt,center]{title}
      \usebeamerfont{title}{\huge\textbf{Statische Typsysteme\\für JavaScript}}
    \end{beamercolorbox}

    \begin{beamercolorbox}[sep=6pt,center]{title}
      \usebeamerfont{title}\large Entwicklung eines Transpilers zur\\Übersetzung von Flow nach TypeScript
    \end{beamercolorbox}

    \vspace{0.5cm}

    \begin{beamercolorbox}[sep=3pt,center]{author}
      \usebeamerfont{author}\normalsize\insertauthor
    \end{beamercolorbox}

    \begin{beamercolorbox}[sep=3pt,center]{institute}
      \usebeamerfont{institute}{\normalsize\insertinstitute}
    \end{beamercolorbox}

    \vspace{0.5cm}

    \begin{beamercolorbox}[sep=3pt,center]{date}
      \usebeamerfont{date}{\normalsize\insertdate}
    \end{beamercolorbox}
  \end{centering}
  \vfill
}
\makeatother

\newcommand{\pie}[1]{%
	\begin{tikzpicture}%
    \draw[pcolor] (0,0) circle (0.7ex);%
    \fill[rotate=-90,fill=pcolor] (0.7ex,0) arc (0:-#1*180:0.7ex) -- (0,0) -- cycle;%
	\end{tikzpicture}%
}

\newcommand*\circled[1]{
  \tikz[baseline=(char.base)]{
    \node[shape=rectangle,fill=white,text=scolor,inner sep=6pt] (char) {#1};
  }
}

\newcommand{\secframe}[2]{
  {
    \setbeamercolor{background canvas}{bg=scolor}
    \begin{frame}[plain]
      \Huge\color{white}\textbf{\circled{#1}\hspace{0.5em}#2}
    \end{frame}
  }
}
