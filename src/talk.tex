\documentclass{beamer}

% Metadata
\title{Statische Typsysteme für JavaScript}
\subtitle{Entwicklung eines Transpilers zur Übersetzung von Flow nach TypeScript}
\author{Jonathan Gruber}
\institute{HTWK Leipzig}
\date{18. Dezember 2019}

% Encoding
\usepackage[utf8]{inputenc}
\usepackage[T1]{fontenc}

% Babel
\usepackage[ngerman]{babel}

% Fonts
\usepackage[sb]{plex-sans}
\usepackage[scale=0.8,sb]{plex-mono}

% Biber (references)
\usepackage[backend=biber]{biblatex}
\addbibresource{src/references.bib}

\let\OLDitemize\itemize
\renewcommand\itemize{\OLDitemize\addtolength{\itemsep}{4pt}}

\let\OLDenumerate\enumerate
\renewcommand\enumerate{\OLDenumerate\addtolength{\itemsep}{2pt}}

\usepackage{csquotes}
\usepackage{longtable, tabu, booktabs}
\usepackage{tikz}
\usepackage{hyperref}

% Code listings
\usepackage{listings}

\lstset{
  aboveskip=0pt,
  basicstyle=\linespread{1.04}\small\ttfamily,
  belowcaptionskip=0pt,
  belowskip=0pt,
  breakatwhitespace=true,
  breaklines=true,
  comment=[l]{//},
  commentstyle=\color{darkgray},
  emph={?, any, Array, boolean, empty, interface, mixed, never, number, null, opaque, string, this, unknown, void, T, TypeAlias},
  emphstyle=\itshape,
  extendedchars=false,
  keywords={abstract, arguments, await, boolean, break, byte, case, catch, char, class, const, continue, debugger, declare, default, delete, do, double, else, enum, eval, export, extends, false, final, finally, float, for, function, goto, if, implements, import, in, instanceof, int, let, long, module, native, new, of, package, private, protected, public, return, short, static, super, switch, synchronized, this, throw, throws, transient, true, try, typeof, var, volatile, while, with, yield, type},
  keywordstyle=\bfseries,
  morecomment=[s]{/*}{*/},
  morestring=[b]',
  morestring=[b]",
  numbers=left,
  numbersep=12pt,
  numberstyle=\linespread{1.04}\scriptsize\ttfamily\color{gray},
  showspaces=false,
  showstringspaces=false,
  stepnumber=1,
  stringstyle=\color{black},
  tabsize=4,
}

% The following ensures that only straight quotes are used inside of \texttt
\usepackage{upquote}
\usepackage{etoolbox}

\robustify{\texttt}
\let\originaltexttt\texttt

\begingroup
\catcode`'=\active
\catcode``=\active
\globaldefs1
\makeatletter
\renewrobustcmd{\texttt}[1]{%
  {%
  \everyeof{\noexpand}\endlinechar-1
  \expandafter\catcode\string``=\active
  \expandafter\catcode\string`'=\active
  \let'\textquotesingle
  \let`\textasciigrave
  \ifx\encodingdefault\upquote@OTone
    \ifx\ttdefault\upquote@cmtt
    \def'{\char13 }\def`{\char18 }%
    \fi
  \fi
  \scantokens{\originaltexttt{#1}}%
  }%
}%
\endgroup


% Theme
\definecolor{pcolor}{HTML}{00695c}
\usetheme{Frankfurt}
\setbeamercolor{bibliography entry author}{fg=black}
\setbeamercolor{bibliography entry location}{fg=black}
\setbeamercolor{bibliography entry note}{fg=black}
\setbeamercolor{bibliography entry title}{fg=black}
\setbeamercolor{enumerate item}{fg=black}
\setbeamercolor{enumerate subitem}{fg=black}
\setbeamercolor{itemize item}{fg=black}
\setbeamercolor{itemize subitem}{fg=black}
\setbeamerfont{frametitle}{series=\bfseries,parent=structure,}
\setbeamertemplate{bibliography item}{\insertbiblabel}
\setbeamertemplate{blocks}[rounded][shadow=false]
\setbeamertemplate{enumerate items}[default]
\setbeamertemplate{frametitle}[default][shadow=false]
\setbeamertemplate{footline}[frame number]
\setbeamertemplate{itemize items}[circle]
\setbeamertemplate{navigation symbols}{}
\setbeamertemplate{title page}[default][rounded=true,shadow=false]
\usecolortheme[named=pcolor]{structure}

% Small font in bibliography
\renewcommand*{\bibfont}{\small}

% Titlepage
\makeatletter
\setbeamertemplate{title page}
{
    \vfill
    \begin{centering}
        \setbeamercolor{title}{bg=white,fg=pcolor}

        \begin{beamercolorbox}[center]{author}
            \usebeamerfont{author}\normalsize Masterkolloquium
        \end{beamercolorbox}

        \vspace{0.5cm}

        \begin{beamercolorbox}[sep=8pt,center]{title}
          \usebeamerfont{title}{\huge\textbf{Statische Typsysteme\\für JavaScript}}
        \end{beamercolorbox}

        \setbeamercolor{title}{bg=white,fg=black}
        \begin{beamercolorbox}[sep=6pt,center]{title}
            \usebeamerfont{title}\large Entwicklung eines Transpilers zur\\Übersetzung von Flow nach TypeScript
        \end{beamercolorbox}

        \vspace{0.5cm}

        \begin{beamercolorbox}[sep=3pt,center]{author}
            \usebeamerfont{author}\normalsize\insertauthor
        \end{beamercolorbox}

        \begin{beamercolorbox}[sep=3pt,center]{institute}
            \usebeamerfont{institute}{\normalsize\insertinstitute}
        \end{beamercolorbox}

        \vspace{0.5cm}

        \begin{beamercolorbox}[sep=3pt,center]{date}
            \usebeamerfont{date}{\normalsize\insertdate}
        \end{beamercolorbox}
    \end{centering}
    \vfill
}
\makeatother

% Footnotes without markers
% \newcommand\blfootnote[1]{%
%     \begingroup
%     \renewcommand\thefootnote{}\footnote{#1}%
%     \addtocounter{footnote}{-1}%
%     \endgroup
% }

\newcommand*\circled[1]{
    \tikz[baseline=(char.base)]{
        \node[shape=circle,fill=pcolor,text=white,inner sep=3pt] (char) {\textbf{#1}};
    }
}

\newcommand{\chaptitle}[2]{\Large\color{pcolor}\textbf{\circled{#1}\hspace{0.4em}#2}}


\begin{document}
  \frame[plain,noframenumbering]{\titlepage}

  \frame[plain,noframenumbering]{
    \frametitle{}
    \LARGE{\textbf{Überblick}}
    \vspace{1em}
    \large
    \begin{enumerate}
      \item Problemstellung
      \item Statische Typsysteme für JavaScript
      \item Umsetzung des Transpilers
      \item Ergebnisse und Auswertung
      \item Fazit
    \end{enumerate}
  }

  \section{Problemstellung}
    \secframe{Problemstellung}

    \begin{frame}
      \frametitle{Motivation}
      \begin{itemize}
        \item JavaScripts \textbf{dynamische Typisierung} erschwert die Entwicklung korrekter, sicherer und wartbarer Anwendungen~\autocite{NIKHIL:2014,PRADEL:2015,BIERMAN:2014}
        \item Typ von Werten wird zur Laufzeit implizit umgewandelt (\textit{type coercion})
        \item Kritisches Laufzeitverhalten wird zum Teil ignoriert
        \item aber: Erkennung von Programmfehlern durch Einsatz eines \textbf{statischen Typsystems} möglich
        \item Zwei Vertreter: \textit{Flow}~\autocite{FLOW:PAPER} und \textit{TypeScript}~\autocite{TYPESCRIPT:SPEC}
      \end{itemize}
    \end{frame}

    \begin{frame}[fragile]
      \frametitle{Beispiel -- Statische Typisierung durch Flow}
      \begin{lstlisting}
function linearSearch<T>(list: Array<T>, searchValue: T): number | null {
  for (const [index, value] of list.entries()) {
    if (value === searchValue) {
      return index;
    }
  }
  return null;
}

linearSearch<number>([3, 56, 10], 56);           // 1
linearSearch<number>([3, 56, 10], 12);           // null
linearSearch<string>(['foo', 'bar', 'baz'], 3);  // Typfehler
      \end{lstlisting}
    \end{frame}

    \frame{
      \frametitle{Zielsetzung der Masterarbeit}
      \begin{itemize}
        \item Ausgangslage: \textit{Spreadshirt} setzt Flow zur statischen Typisierung von JavaScript-Anwendungen ein
        \item Ziel: Migration bestehender Flow"=Projekte nach TypeScript
        \item Händische Migration fehleranfällig und zeitaufwändig
        \item Lösung: Entwicklung eines \textbf{Transpilers} zur Übersetzung der Flow-Typisierung nach TypeScript
        \item Transpiler = Compiler, der Quelltext einer Programmier"=sprache in eine andere übersetzt
      \end{itemize}
    }

    \begin{frame}[fragile]
      \frametitle{Beispiel einer Typ-Übersetzung}

      \begin{lstlisting}[
        numbers=none,
        basicstyle=\linespread{1.04}\large\ttfamily,
        emph={number,null,undefined},
      ]
// Flow
type MaybeNumber = ?number;

// TypeScript
type MaybeNumber = number | null | undefined;
      \end{lstlisting}
    \end{frame}

    \begin{frame}
      \frametitle{Ziele der Migration zu TypeScript}
      \textbf{Z1}\hspace{0.75em}Erkennung weiterer Typ- und Programmfehler\\[.6em]
      \textbf{Z2}\hspace{0.75em}Unterstützung externer Bibliotheken\\[.6em]
      \textbf{Z3}\hspace{0.75em}Performance der Typüberprüfungen\\[.6em]
      \textbf{Z4}\hspace{0.75em}Zukunftssicherheit und Transparenz der Technologie
    \end{frame}

  \section{Statische Typsysteme}
  \secframe{Statische Typsysteme\secframebr für JavaScript}

    \begin{frame}
      \frametitle{Flow}
      \begin{itemize}
        \item ermöglicht Verwendung von Typannotationen in\\regulären JavaScript-Quelltexten
        \item Typüberprüfung durch Server im Hintergrund
        \item Fokus auf Korrektheit\footnote{engl. \textit{Soundness}} des Typsystems
        \item globale Typinferenz durch pfadsensitive Datenflussanalyse
        \item Typsystem größtenteils strukturell, z. T. nominal
        \item Parallelisierung der Berechnung
      \end{itemize}
    \end{frame}

    \begin{frame}
      \frametitle{TypeScript}
      \begin{itemize}
        \item vollständige Programmiersprache mit\\statischer Typisierung
        \item Syntax ist strikte Obermenge von JavaScript
        \item Typüberprüfung und Übersetzung nach JavaScript\\durch TypeScript-Compiler
        \item \enquote{\textit{Balance zwischen Korrektheit und Produktivität}}~\autocite{TYPESCRIPT:DESIGN_GOALS}
        \item Typinferenz nur lokal und teilweise kontextuell
        \item Typsystem rein strukturell
        \item graduelle Typisierung möglich
      \end{itemize}
    \end{frame}

  \section{Umsetzung}
    \secframe{Umsetzung des\secframebr Transpilers}

    \begin{frame}
      \frametitle{Technische Anforderungen an den Transpiler}
      \textbf{A1}\hspace{0.75em}Äquivalente und vollständige Übersetzung der Flow-Typen\\[.6em]
      \textbf{A2}\hspace{0.75em}Beibehaltung der Programmsemantik\\[.6em]
      \textbf{A3}\hspace{0.75em}Unterstützung aktueller und vorläufiger JavaScript- sowie\\\hspace{2.05em}JSX-Syntax\\[.6em]
      \textbf{A4}\hspace{0.75em}Verarbeitung gesamter Projektverzeichnisse\\[.6em]
      \textbf{A5}\hspace{0.75em}Beibehaltung der Quelltextformatierung
    \end{frame}

    \begin{frame}
      \frametitle{Implementierung als Babel-Plugin}
      \begin{itemize}
        \item \textit{Babel}~\autocite{BABEL} als Grundlage der Implementierung gewählt
        \item Babel = Transpiler für JavaScript
        \item unterstützt alle geforderten Arten von Syntax
        \item gute Erweiterbarkeit durch Plugin-System
        \item sehr ausgereiftes Projekt
      \end{itemize}
    \end{frame}

    \begin{frame}
      \frametitle{Programmtransformation durch Babel}
      \textbf{\large Grundkonzept}
      \vspace{1em}
      \begin{enumerate}
        \item Parsen des Eingabequelltexts\linebreak$\Rightarrow$ abstrakter Syntaxbaum (AST) des Programms\\
        \smallskip
        \item \textbf{Programmtransformation} durch Manipulation der AST-Knoten mittels \textit{Besucher-Entwurfsmuster}\\
        \smallskip
        \item Generierung des Ausgabequelltexts
      \end{enumerate}
    \end{frame}

    \begin{frame}[fragile]
      \frametitle{Transformation des abstrakten Syntaxbaums}

      \begin{columns}
        \begin{column}{.48\textwidth}
          \begin{lstlisting}[numbers=none]
// @flow
type Alias = mixed;
          \end{lstlisting}
          \vspace{5mm}
          \ttfamily
          \begin{forest}
            for tree = {l=1.6cm, s sep=0.4cm}
            [Program
              [TypeAlias, edge label={node[edge]{body}}
                [Identifier, edge label={node[edge]{id}}
                  [\enquote{Alias}, edge label={node[edge]{name}}]
                ]
                [MixedTypeAnnotation, edge label={node[edge]{right}}]
              ]
            ]
          \end{forest}
        \end{column}
        \begin{column}{.04\textwidth}

        \end{column}
        \begin{column}{.48\textwidth}
          \begin{lstlisting}[numbers=none]
  // TypeScript
  type Alias = unknown;
          \end{lstlisting}
          \vspace{5mm}
          \ttfamily
          \begin{forest}
            for tree = {l=1.6cm, s sep=0.5cm}
            [Program
              [TSTypeAliasDeclaration, edge label={node[edge]{body}}
                [Identifier, edge label={node[edge]{id}}
                  [\enquote{Alias}, edge label={node[edge]{name}}]
                ]
                [TSUnknownKeyword, edge label={node[edge]{typeAnnotation}}]
              ]
            ]
          \end{forest}
        \end{column}
      \end{columns}
    \end{frame}

    \begin{frame}
      \vspace{1mm}
      \begin{columns}
        \column{\dimexpr\paperwidth-8mm}
        \begin{figure}
          \includegraphics[width=\textwidth]{src/figures/architecture-overview.pdf}
        \end{figure}
      \end{columns}
    \end{frame}

    % \begin{frame}
    %   \frametitle{Spezialfälle}
    %   TODO: Spezialfälle der Transformation
    % \end{frame}

  \section{Ergebnisse}
    \secframe{Ergebnisse und\secframebr Auswertung}

    % \begin{frame}
    %   \frametitle{Durchführung der Migration}
    %   \begin{itemize}
    %     \item Zwei Projekte von Spreadshirt wurden mit Hilfe des Transpilers erfolgreich migriert
    %     \item aber: daraufhin zahlreiche neue Typfehler
    %     \item manuelle Korrektur dieser Fehler notwendig
    %   \end{itemize}
    % \end{frame}

    \subsection{Technische Anforderungen}
      % \begin{frame}
      %   \LARGE\textbf{Erfüllung der technischen Anforderungen}
      % \end{frame}

      \begin{frame}
        \frametitle{A1\hspace{0.75em}Äquivalenz und Vollständigkeit der Transf. (1)}
        \begin{block}{Äquivalenz der Übersetzungen}
          \begin{itemize}
            \item wurde größtenteils erzielt
            \item aber: einzelne Funktionen von Flow werden in TypeScript nicht unterstützt\\
              \smallskip
              $\Rightarrow$ Verlust von Typinformation
            \item für 3 der 17 Hilfstypen von Flow konnte kein äquivalenter TypeScript-Ausdruck gefunden werden\\
              \smallskip
              $\Rightarrow$ Ersetzung mit \textit{any}
          \end{itemize}
        \end{block}
      \end{frame}

      \begin{frame}
        \frametitle{A1\hspace{0.75em}Äquivalenz und Vollständigkeit der Transf. (2)}
        \begin{block}{Vollständigkeit der Transformationen}
          \begin{itemize}
            \item wurde erzielt
            \item Umfang der \textit{Vollständigkeit} durch AST-Spezifikation~\autocite{BABEL:PARSER_SPEC} von Babel präzise eingegrenzt
            \item Verifizierung durch 1022 Fixture-Tests, die alle Funktionen von Flow abbilden
            \item außerdem: Überprüfung der Implementierung durch statische Typisierung von Babel
          \end{itemize}
        \end{block}
      \end{frame}

      \begin{frame}
        \frametitle{A2\hspace{0.75em}Beibehaltung der Programmsemantik}
          \begin{itemize}
            \item Voraussetzung: Transpiler adressiert lediglich Flow-Typen und hat keine unbeabsichtigte Nebenwirkung
            \item Verifizierung durch Fixture-Tests
            \item Überprüfung der Beibehaltung der Semantik durch 373 bzw. 326 Modultests der migrierten Projekte\\
              \smallskip
              $\Rightarrow$ keine Fehler
            \item aber: Testabdeckung bei 90,5\% bzw. lediglich 18,0\%
            \item Erfüllung der Anforderung wird angenommen
          \end{itemize}
      \end{frame}

      \begin{frame}
        \frametitle{A3\hspace{0.75em}Unterstützung von JavaScript- und JSX-Syntax}
      \end{frame}

      \begin{frame}
        \frametitle{A4\hspace{0.75em}Verarbeitung gesamter Projektverzeichnisse}
      \end{frame}

      \begin{frame}
        \frametitle{A5\hspace{0.75em}Beibehaltung der Quelltextformatierung}
        \begin{itemize}
          \item ursprüngliche Quelltextformatierung geht durch Transpilierung verloren
          \item deshalb: Formatierungsroutine auf Basis von\\\textit{Prettier}~\autocite{SOFTWARE:PRETTIER} implementiert
          \item Manuelle Überprüfung der originalgetreuen\\Formatierung der Ausgabe in zwei Projekten:\\
            \medskip
            \hspace{1em}Projekt A:~~3,5\% Fehlerrate\\
            \smallskip
            \hspace{1em}Projekt B:~~5,9\% Fehlerrate
          \item Somit: vereinzelt fehlerhafte Formatierung, aber insgesamt zu 95,3\% korrekt
        \end{itemize}
      \end{frame}

    \subsection{Ziele}
      % \begin{frame}
      %   \LARGE\textbf{Erfüllung der Zielvorgaben}
      % \end{frame}

      \begin{frame}
        \frametitle{Z1\hspace{0.75em}Erkennung weiterer Typ- und Programmfehler}
      \end{frame}

      \begin{frame}
        \frametitle{Z2\hspace{0.75em}Unterstützung externer Bibliotheken}
      \end{frame}

      \begin{frame}
        \frametitle{Z3\hspace{0.75em}Performance der Typüberprüfungen}
      \end{frame}

      \begin{frame}
        \frametitle{Z4\hspace{0.75em}Zukunftssicherheit und Transparenz}
      \end{frame}

  \section{Fazit}
    \secframe{Fazit}

    \begin{frame}
      \frametitle{Fazit}

      \begin{block}{Erfüllung der Ziele}
        \begin{enumerate}
          \item Erkennung weiterer Typ- und Programmfehler
          \item Unterstützung externer Bibliotheken
          \item Performance der Typüberprüfungen
          \item Zukunftssicherheit und Transparenz der Technologie
        \end{enumerate}
      \end{block}
    \end{frame}

  \appendix

    \begin{frame}[noframenumbering,plain]
      % empty
    \end{frame}

  \section{Backup}
    \setbeamertemplate{footline}{}

    \begin{frame}[noframenumbering]
      \frametitle{Evaluation bestehender Werkzeuge}
      {
  \footnotesize
  \begin{tabu} to \textwidth {@{}lllccccccrrX@{}}
    \midrule
    Werkzeug & Typ & Format & Erw. & ES10 & ES10+ & Flow & TS & JSX \\
    \midrule
    Acorn     & P  &  ESTree  & \pie{2} & \pie{2} & \pie{1} & \pie{0} & \pie{0} & \pie{2} \\ % 2012
    Astring   & G  &  ESTree  & \pie{2} & \pie{2} & \pie{1} & \pie{0} & \pie{0} & \pie{0} \\ % 2015
    Babel     & PG &  Babel   & \pie{2} & \pie{2} & \pie{2} & \pie{2} & \pie{2} & \pie{2} \\ % 2014
    Escodegen & G  &  ESTree  & \pie{0} & \pie{0} & \pie{0} & \pie{0} & \pie{0} & \pie{0} \\ % 2012
    Esprima   & P  &  ESTree  & \pie{0} & \pie{1} & \pie{0} & \pie{0} & \pie{0} & \pie{1} \\ % 2011
    Recast    & PG &  diverse & \pie{2} & \pie{2} & \pie{2} & \pie{2} & \pie{2} & \pie{2} \\ % 2012
    \midrule
  \end{tabu}

  \vspace{5mm}
  \begin{columns}[T]
    \begin{column}{0.4\textwidth}
      {
        \renewcommand{\arraystretch}{1.1}
        \begin{tabular}{@{}ll@{}}
          P & Parser\\
          G & Codegenerator\\
          \pie{0} & keine Unterstützung\\
          \pie{1} & teilweise Unterstützung\\
          \pie{2} & vollständige Unterstützung\\
        \end{tabular}
      }
    \end{column}
    \begin{column}{0.5\textwidth}
      {
        \renewcommand{\arraystretch}{1.1}
        \begin{tabular}{@{}ll@{}}
          ES10 & ECMAScript 2019\\
          ES10+ & vorgeschlagene Spracherw.\\
          TS & TypeScript\\
        \end{tabular}
      }
    \end{column}
  \end{columns}
}

    \end{frame}

    \begin{frame}
      \frametitle{Nicht äquivalent übersetzbare Flow-Funktionen}
      \begin{itemize}
        \item \textit{Existential type}
        \item Beliebige Typen in Index-Signaturen von Objekttypen
        \item \textit{Opaque type}
        \item Expliziter Rückgabewert von Konstruktoren
        \item Varianz\footnote{TypeScript unterstützt nur Kovarianz}
      \end{itemize}
    \end{frame}

    \begin{frame}[plain,allowframebreaks,noframenumbering]
      \frametitle{Quellenverzeichnis}
      \printbibliography
    \end{frame}
\end{document}
